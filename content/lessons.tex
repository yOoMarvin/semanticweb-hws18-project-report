\section{Lessons Learned}
Besides the overall satisfaction with the development and the results of the project, there were still some obstacles and challenges during development that had to be faced.
\\

One of the major challenges was to find relevant data sources that provide enough data that could be used for the clustering. Besides that, the integration of the data turned out to be hard. Because of the different formats a lot of the available data had to be re-formatted for a simpler usage.
\\

As mentioned before the development team implemented the major parts of the different application areas independently. In the end, all results had to be brought together and it had to be ensured that every part works seemingly together with the rest of the application. This was especially difficult for dynamic user inputs. If the user searches for a specific city this input has to be routed to the backend server where the data is dynamically fetched, prepared and afterward clustered. The final result has to be brought into a JSON format and routed back to the application frontend where the result is displayed. This processing pipeline contains complex, individual steps and a lot of data cleaning is needed to ensure the correct functionality. Additionally, a lot of different techniques and technologies were used in the project, so that common standards for development and communication had to be found. A unified, more homogenous tech stack would make this challenge a lot easier.
\\

All in all the project was overall successful and the achieved results are satisfactory and fulfill the predefined application goal. The application of Semaps brings a semantic web approach to maps and makes it easier to get to know a city by simply looking at the map.